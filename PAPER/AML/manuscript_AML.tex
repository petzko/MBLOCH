\documentclass[journal]{IEEEtran}

\renewcommand{\baselinestretch}{1.0} % Change to 1.65 for double spacing

\usepackage{amsmath,amsfonts,amssymb}
\usepackage{graphicx}
\usepackage[colorlinks=true, allcolors=blue]{hyperref}


\usepackage{color}
\usepackage[latin9]{inputenc}
\usepackage{mathrsfs,amsmath}
\usepackage{graphicx}%
\usepackage{float}
\usepackage{amsfonts}%
\usepackage{amssymb}
\usepackage{braket}
\usepackage{bm}
\usepackage{cite}
%necessary for outline section of the article 
%\usepackage{outline}

\newcommand{\mb}[1]{\bm{#1}}
\usepackage[T1]{fontenc}
\def\Nabla{\bm{\nabla}}
\def\bm{\mathbf}
\def\curl{\Nabla\times}
\def\div{\Nabla\cdot}
\def\lap{\Delta}
\def\vlap{\Delta}
\def\x{\hat{e}_{x}}
\def\y{\hat{e}_{y}}
\def\z{\hat{e}_{z}}
\def\p{\partial}
\def\h{\hat}
\def\tw{\tilde{\omega}}
\def\gm{\gamma}
\def\om{\omega}
\def\OM{\Omega}
\def\GM{\Gamma}
\def\dw{\delta\omega}
\def\dth{\Delta\theta}
\def\dk{\delta k}
\def\Hdth{\frac{\dth}{2}} %half Delta Theta
\DeclareMathOperator{\Tr}{Tr}

\newcommand{\includegraphicsL}[1]{\includegraphics[width=0.80\textwidth]{#1}}
\newcommand{\includegraphicsS}[1]{\includegraphics[width=0.40\textwidth]{#1}}
\newcommand{\includegraphicsXS}[1]{\includegraphics[width=0.35\textwidth]{#1}}
\newcommand{\includegraphicsBIO}[1]{\includegraphics[width=1.1in,height=1.1in,clip,keepaspectratio]{#1}}
\title{Coupled transmission line/Maxwell-Bloch equations approach for electro-optical simulations of terahertz quantum cascade lasers}
\author{\IEEEauthorblockN{
		Petar Tzenov\IEEEauthorrefmark{1},
		Longway Zhong\IEEEauthorrefmark{1},
		David Burghoff\IEEEauthorrefmark{2},
		Qing Hu\IEEEauthorrefmark{2}, 
		Christian Jirauschek\IEEEauthorrefmark{1}}
	
	\IEEEauthorblockA{\IEEEauthorrefmark{1}Institute for Nanoelectronics, Technical University of Munich, D-80333 Munich, Germany}
	
	\IEEEauthorblockA{\IEEEauthorrefmark{2}Department of Electrical Engineering and Computer Science, Research Laboratory of Electronics, Massachusetts Institute of Technology, Cambridge, Massachusetts 02139, USA}
	\thanks{Corresponding author: P. Tzenov (email: petar.tzenov@tum.de).}}



\IEEEtitleabstractindextext

\begin{document} 
	\maketitle
		
\begin{abstract}
Active mode locking (AML) via modulation of the injection current or bias is a standard technique employed for the generation of ultrashort pulses in electrically pumped lasers. Quantum cascade lasers (QCLs), as sources of radiation in the mid- and far-infrared portions of the electromagnetic spectrum, have turned out to be exceedingly difficult to actively mode lock, due to the inherently short gain recovery time of these kind of devices. In the mid-infrared, both theoretical and experimental results have shown that this obstacle can be overcome by modulating only a short, electrically isolated section of the QCL cavity, which could lead to generation of ultrashort picosecond pulses. For terahertz (THz) QCLs, most recently successful active mode locking of an LO-phonon THz-QCL was reported, and pulses as short as 11 ps were detected. Furthermore, in the same work, the importance of correct coupling between the propagating gigahertz (GHz) and terahertz fields was explicitly outlined and the role of the wave-guiding structure in the modulation process emphasized. Here, we present a theoretical model based on the Maxwell-Bloch and the transmission line equations, suitable for investigation of such systems. (OLD ABSTRACT TODO REWRITE!)
\end{abstract}



\section{Theoretical model}
The ever increasing complexity of chip-scale devices necessitates correspondingly sophisticated theoretical modelling. In the last decade, we have seen a steady progress in the design of quantum cascade lasers, resulting in devices with high electrical stability [cite linewidth papers], emitting spectra with various desirable characteristics such as high-power single mode emission [cite], short pulse generation [cite] and frequency comb emission [cite].

On the contrary it seems that theoretical or simulation models have struggled to keep up with the pace of progress in the field. This could be partly explained due to the complicated dynamics of QCLs, stemming from the intricate interplay various coherent and incohrent processes [cite], nonlinear light-matter interactions and complicated electro-optical phenomena []. Simulation approaches with various degrees of complexity have been proposed XXX

\subsection{Optical modelling}
To model the active regions we adapt the density matrix model from Ref. [cite Belyanin], to include total of four subband levels, and more importantly to allow all relevant system parameters, such as the eigenenergies, dipole moments, scattering rates etc. to be bias dependent. The time evolution is thus governed by a four level Maxwell-Bloch equations, coupled ot a wave propagation equation for the $z$-component electric field $E_z(x,t)$, which is also the assumed growth direction of the QCL. 
\begin{subequations}
	\label{eq:dmequations}
 \begin{align}
	\frac{d\rho_{44}}{dt} &= J(x,t)   +i\frac{ez_{43}}{\hbar}E_z(\rho_{43}-\rho_{34}) + \sum_{j\neq 4} \frac{\rho_{jj}}{\tau_{j\rightarrow 4}}  - \frac{\rho_{44}}{\tau_{2}}, \\
	\frac{d\rho_{33}}{dt} &= -i\frac{ez_{43}}{\hbar}E_z(\rho_{43}-\rho_{34}) + \sum_{j\neq 3} \frac{\rho_{jj}}{\tau_{j\rightarrow 3}} - \frac{\rho_{33}}{\tau_{3}}, \\
	\frac{d\rho_{22}}{dt} &= \sum_{j\neq 2} \frac{\rho_{jj}}{\tau_{j\rightarrow 2}} - \frac{\rho_{22}}{\tau_{2}}, \\
	\frac{d\rho_{11}}{dt} &=  -J(x,t) + \sum_{j\neq 1} \frac{\rho_{jj}}{\tau_{j\rightarrow 1}} - \frac{\rho_{11}}{\tau_{1}}, \\
	\frac{d\rho_{43}}{dt} &= - i\omega_{43}\rho_{43} + i\frac{ez_{43}}{\hbar}E_z(\rho_{44}-\rho_{33})-\Gamma_{\parallel 43}\rho_{43}.
 \end{align}
\end{subequations}
In Eqs. (\ref{eq:dmequations}) $\rho_{ij}$ denotes the $ij-$th density matrix element,  $z_{43}$ the optical transition's dipole moment, $e$ the elementary charge, $\hbar$ is the reduced Plank's constant. Furthermore, since our model assumes position and time dependent bias $V(x,t)$, this in turn results in spatio-temporal  dependence of the transition frequency $\omega_{43}(x,t)$, the scattering rates $1/\tau_{ij}(x,t)$, the dephasing rate $\Gamma_{\parallel 43}(x,t)$ as well as current density $J(x,t)$. Finally the parameters $1/\tau_{j}(x,t)$ denote the lifetimes of the corresponding levels and $J(x,t)$ is the second order tunneling current from the injector state to the upper laser level and is given by the formula [cite]
\begin{align}
\label{eq:current}
J &= en^s\frac{\Omega_{AC}^22\Gamma_{\parallel 1'4}}{\epsilon^2+4\Gamma_{\parallel 1'4}^2}\Big\{\Theta(\epsilon)(\rho_{11}-\rho_{44}e^{-|\hbar\epsilon|/k_BT}) \nonumber \\
&+\Theta(-\epsilon)(\rho_{11}e^{-|\hbar\epsilon|/k_BT}-\rho_{44})\Big\}.
\end{align}
In Eq. (\ref{eq:current}) $n^s$ denotes the sheet carrier density, $\Theta(\cdot)$ is the Heaviside function and the term $e^{-|\hbar\epsilon|/k_BT}$ denotes an effective "weight" factor modelling the assumption of thermalized $k$-space distribution of the injector and upper laser level electrons with the same thermal energy $k_BT$ in each subband.   
\begin{figure}[H]
  	\label{fig:WFs}
  	\centering
  	\includegraphicsS{IMGS/WFs}
  	\caption{insert caption here.}	
\end{figure}
Further 

\subsection{Electrical modelling}
This is the electrical geometry -> TL equations write them here
\begin{figure}[H]
	\label{fig:circuit}
	\centering
	\includegraphicsS{IMGS/CIRCUIT}
	\caption{insert caption here.}	
\end{figure}

 

\subsection{Coupled model}
\begin{figure*}[tb]
	\label{fig:coupledmodel}
	\centering
	\includegraphicsL{IMGS/coupledmodel}
	\caption{insert caption here.}	
\end{figure*}
We propose a combined solution to the above enumerated issues in the form of a coupled ensemble Monte Carlo/Transmission Line equations/Maxwell-Bloch simulation approach. The whole algorithm is described in Fig. [].

We start with our Schr�dinger-Poisson and EMC simulation codes and calculate the bias dependence of the various scattering mechanisms, eigenenergies, tunneling coupling strengths (anticrossing parameters) as well as the dipole moments. Even though this procedure by itself is quite time-consuming, it needs to be performed only once and the results are stored for further processing. Also, one can achieve significant speedup of the computations with the aid of modern parallelization approaches, utilizing multi-core simulation servers or larger compute clusters. 

The bias grid was chosen to span sufficiently broad interval, i.e. between 8.0 kV/cm and 14.0 kV/cm, with a step of 0.2 kV/cm. The bias dependence of the most critical parameters, and namely the various scattering rates, the eigenenergies of the bound states, the dipole moment of the optical transition as well as the tunneling coupling coefficient are illustrated in Fig. \ref{fig:params_v}.
\begin{figure}[H]
	\label{fig:params_v}
  	\centering
  	\includegraphicsS{IMGS/params_v}
  	\caption{insert caption here.}	
\end{figure}

\begin{enumerate}
	\item Illustrate the general coupling scheme -> Schr�dinger Poisson Solver-> ensemble Monte Carlo -> Maxwell-Bloch and Transmission-Line equations  
	\item simulation scheme as a block-diagram. 
	\item Illustrate the laser under study -> Optica laser -allows direct comparison with experiment. \cite{petz2016}  
\end{enumerate}


\section{Simulation results and comparison with experiment}
\subsection{IVL characteristics}
\subsection{Injection locking}
\subsection{Active mode locking and ultrashort pulse generation}

\begin{appendices}
\section{Numerical methods}
\section{Boundary conditions}
\section{Simulation parameters}
\end{appendices}

\section*{Funding}
This work was supported by the German Research Foundation (DFG) within the Heisenberg program (JI 115/4-1) and under DFG Grant No. JI 115/9-1.

\bibliographystyle{IEEEtran}
\bibliography{literature/AML_bib_resources.bib}


\end{document} 
