%
% Titel-tex
%

% Die vertikalen Abstände nach Eingabe der Daten bitte selbst anpassen.

\begin{titlepage}

\begin{center}
    \includegraphics[width=\textwidth]{Logos2.pdf}% [scale=0.7, right]
    
    \vspace{25mm}	
    
    {
        \scshape	
        Technische Universität München\\
        Fakultät für Elektrotechnik und Informationstechnik\\																										% 
        Professur für Computational Photonics\\
        Prof.Dr.-Ing. Christian Jirauschek
    }
		
    \vspace{25mm}																								

    \begin{LARGE}
        \normalfont\rmfamily \textbf{Coupled Transmission Line/Maxwell-Bloch Equations Approach for Electro-Optical Simulations of Terahertz Quantum Cascade Lasers}\\[6cm]
    \end{LARGE}
   \so{\Large MASTER THESIS}\\[0.5cm]
    
    by Longwei Zhong

    \textbf{April, 2017}\\[2cm]
																	
 																	
\end{center}

\end{titlepage}

\clearpage \thispagestyle{empty} \cleardoublepage 

%\vspace*{10cm}

%\begin{center}
%\it To my family and friends
%\end{center}


%\thispagestyle{empty} \cleardoublepage 


%%Deckblatt farbig
%
%\makeatletter		%braucht man weil der nachfolgende Befehl ein @ enthält und nur im Kernel ausgeführt werden kann, oder in einem package
% \@colht\paperheight% das ist das Äquivalent zu \textheight, das man aber nur im header verwenden kann.
%\makeatother
%
%
%\begin{titlepage}
%
%\begin{changemargin}{-1cm}{-2cm}
%
%\begin{center}
%	\Huge \textsc{Physik Department} \\
%	%\vspace{0,5cm}
%	\includegraphics[scale=1.8]{Images/physik_blau}
%
%	\vspace{0.5cm}	
%	
%	\textbf{Bli Bla Blubb} \\
%	
%	\vspace{3.5cm}
%	
%	\Large Master Thesis \\
% 		by \\
% 		Ulrich Schwarzenberger \\
% 		
%	\vspace{2.5cm}
%	
%	\large October 2011 \\
%	
%	\vspace{2.7cm}
%	
%	\begin{tabular}{rcl}
%  	\multirow{5}{*}{\includegraphics[width=2.3cm]{Images/wsi_logo_neu}} & Walter Schottky Institut & 			\multirow{5}{*}{\includegraphics[width=2.3cm]{Images/tum-logo}} \\
%  	& Zentralinstitut der Technischen Universität München & \\
%  	& für physikalische Grundlagen der Halbleiterelektronik & \\
%  	& Prof. Dr. Martin Stutzmann & \\
%  \end{tabular}
%  
%\end{center}
%\normalsize
%\vfill
%\end{changemargin}
%
%\end{titlepage}
%
%\clearpage \thispagestyle{empty} \cleardoublepage 